%%
% NCHU Bachelor Proposal Report Template
%
% 南昌航空大学毕业设计开题报告(研究内容及实验方案)—— 使用 XeLaTeX 编译
%
% Copyright 2023 Arnold Chow
%
% The Current Maintainer of this work is Arnold Chow.
%
% Compile with: xelatex -> biber -> xelatex -> xelatex

\section{研究内容及实验方案}
% 要求尽量详尽,应有500字以上。
% 研究内容:
% \begin{enumerate}[label=\arabic*)]
%     \item 要实现研究的子目标,分别要对哪些方面进行深入研究?这些研究对实现总目标分别起什么样的作用?
%     \item 研究的特色和新意是什么?拓展理论、革新技术、完善方法、解决争议、验证假说等?
% \end{enumerate}

此项研究旨在探索新的图像处理方法,通过组合现有的处理方法探索创新的可能性。
为了实现总体目标则首先需要对以下子目标进行深入研究:
\begin{enumerate}[label=\arabic*)]
    \item 对信号处理基础知识进行深入研究,此举旨在为实际开发新的图像处理算法时提供理论基础。
    \item 研究所面对的问题,分析评估待处理的图像具有的特征以及难点,此举旨在可以指明开发新的图像处理方法的方向。
    \item 了解现有的图像处理方法,以此可以进一步确定研究方向。
    \item 对开发环境进行综合比较,找到适合图像处理的开发环境将对总目标的实现起到事半功倍的效果。
    \item 深入研究相关算法设计方法和原理,其直接关系到算法的性能以及实际处理图像的效果。
\end{enumerate}

% 研究方案:
% \begin{enumerate}[label=\arabic*)]
%     \item 实现研究目标的总体方案是什么?要完成总体方案,必须完成哪几项工作,分别起到什么作用?
%     \item 各项工作中需要解决的关键问题是什么?用什么关键技术来解决这些关键问题?
%     \item 用系统结构图/流程图/逻辑图等方式对技术路线加以说明。
% \end{enumerate}

针对以上研究内容,制定以下研究方案:
\begin{enumerate}[label=\arabic*)]
    \item 研究适合目标图像的图像预处理方法。根据现有的图像预处理对目标图像进行预处理并相互比较,选出最适合目标图像的预处理方法。
    \item 对现有的图像处理方法进行综合判断,从中选择符合自身能力的方法进行研究。
    \item 确定开发环境。将利用Python进行相关程序编辑。
    \item 研究现有的图像处理方法。从网上开源的程序代码中研究其图像处理原理。
    \item 尝试从已有的方法中进行创新。实际开发时将会结合已有的方法进行组合创新,并测试其性能。
\end{enumerate}

% 同时,还需要对困难进行评估:
% \begin{enumerate}[label=\arabic*)]
%     \item 在开展研究时,技术上是否存在障碍,有没有办法克服?
%     \item 如果研究方案出现行不通的情况,是否有备用方案?
% \end{enumerate}

根据研究方向以及研究方案进行综合评估后,预计将会面临以下困难:
\begin{enumerate}[label=\arabic*)]
    \item 在实际开发时面临将图像转换为频域表示时将会出现出现图像损失导致最后图像处理效果变差。
    \item 对现有的图像处理方法了解不足,将会导致出现研究缓慢的情况。
    \item 对于编程语言不熟悉也会导致研究进步缓慢。
    \item 实验结果可能因为程序相关参数设定导致实验结果不理想。
    \item 将已有方法进行组合时将会面临不同方法之间出现冲突。
\end{enumerate}

根据面临的困难提出相对应的解决方法:
\begin{enumerate}[label=\arabic*)]
    \item 在进行图像转换前进行图像填充,以达到在转换后不损失图像。
    \item 寻找图像处理方法相关论文进行学习,总结优缺点后再选择研究方向。
    \item 针对所需的编程语言进行系统性的学习,并积极在网上寻找相关教程。
    \item 针对程序反复进行实验,了解其参数带来的效果,并针对目标图像进行设定。
    \item 尝试多种组合方式并调整相关参数,以达到预期效果。
\end{enumerate}