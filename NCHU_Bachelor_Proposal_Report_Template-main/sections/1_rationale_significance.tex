%%
% NCHU Bachelor Proposal Report Template
%
% 南昌航空大学毕业设计开题报告(选题的依据与意义)—— 使用 XeLaTeX 编译
%
% Copyright 2023 Arnold Chow
%
% The Current Maintainer of this work is Arnold Chow.
%
% Compile with: xelatex -> biber -> xelatex -> xelatex

\section{选题的依据及意义}

近年来,图像处理领域取得了重大进步,在医学、安全和娱乐等领域得到了广泛的应用。虽然现有的图像技术已经较为成熟,
但仍然存在局限性,例如,它们不能很好地处理图像中的纹理和形状等特征。尤其是在缺陷检测时,如果图像在发送
之前不进行处理,一是会导致图像中存在噪声和失真,从而影响检测准确度。二是会过多占据信道带宽,尤其在信号带宽有
限的情况下,这会导致图像的传输速度变慢、产生码间干扰等不良影响。
因此,对图像在发送前进行预处理是必不可少的。在图像处理中,滤波是一种常用的预处理方法,它可以去除图像中的噪声
和失真,从而提高图像的质量。而基带信号传输滤波理论为分析原始形式的信号提供了不同于以往图像处理的方法。与现有
方法相比,这些方法更精确,噪声量更小。此外,基带信号传输滤波理论可用于分析不同图像特征之间的关系,因此可用于
开发新的图像处理算法,以更准确地检测和保留这些特征。所以将基带信号传输滤波理论扩展到图像处理有望为图像处理和
分析提供新的和改进的方法,通过在这一领域进行研究,将有可能进一步了解如何有效和高效地处理图像,从而在使图像处
理的各个领域取得进一步进展。

将基带信号传输滤波理论扩展到图像处理的研究目标是开发新的图像处理和分析方法,而为了实现此目标,则需要了解基带
信号传输滤波理论,并通过研究所需解决的问题有针对性地开发图像处理算法,这涉及确定现有方法的具体局限性
和新算法的所需特征。然后基于对问题的理解,选择合适的理论框架来设计算法。并针对性能和效率进行优化。最后通过模
拟和实验来评估新过滤算法的性能。

此研究预期成果是探索出新的图像滤波方法,以达到尽可能保留图像特征的情况下减少图像传输过程中所占带宽。
% \textbf{前言: \LaTeX 的简单使用:}

% 在这里插入一个参考文献,仅作参考\cite{yuFeiJiZongTiDuoXueKeSheJiYouHuaDeXianZhuangYuFaZhanFangXiang2008}。
% 可以通过空一行(两次回车)实现段落换行,仅仅是回车并不会产生新的段落。

% 正文……\cite{simonyanVeryDeepConvolutional2015}

% 常使用的其他字体格式:

% {\songti \bfseries 宋体加粗} {\textbf{English}}

% {\songti \itshape 宋体斜体} {\textit{English}}

% {\songti \bfseries \itshape 宋体粗斜体} {\textbf{\textit{English}}}

% \textbf{注意!}

% 本模板必须使用 XeLaTeX + BibTeX 编译,否则会直接报错。
% 本模板支持多个平台,结合 Sublime Text/VSCode/Overleaf 都可以使用。
% 本模板支持中英文混排,但是中文和英文之间必须有空格,否则会报错。

% \vspace{10mm}

% 该部分主要需要写:
% \begin{enumerate}[label=\arabic*)]
%     \item 研究课题的名称是什么?为什么要进行该课题的研究?
%     \item 研究的目标是什么?为实现总目标,需要实现哪些子目标?
%     \item 研究的预期成果是什么?该成果具有什么科学价值?
% \end{enumerate}
