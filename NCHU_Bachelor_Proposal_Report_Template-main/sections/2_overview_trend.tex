%%
% NCHU Bachelor Proposal Report Template
%
% 南昌航空大学毕业设计开题报告(国内外研究概况和发展趋势)—— 使用 XeLaTeX 编译
%
% Copyright 2023 Arnold Chow
%
% The Current Maintainer of this work is Arnold Chow.
%
% Compile with: xelatex -> biber -> xelatex -> xelatex

\section{国内外研究概况及发展趋势}
国内外关于图像滤波的研究一直是一个发展非常活跃的领域,图像滤波是图像处理中的一项关键操作,
通常在对图像进行处理时都会先对图像进行滤波,已达到消除噪声和其他伪影来提高图像质量的目的。
它具有广泛的应用,例如图像去噪、边缘检测和图像分割。因此图像滤波的方式非常多,根据不同的
处理要求有相对应的过滤技术。例如常用的高斯滤波器和中值滤波器等,这些滤波器在图像滤波上十
分有效,但因对图像的要求不断提高,这些滤波器已经达不到人们对图像的处理需求,因此各国研究员
根据不同的图像处理需求,开发出了相应的高效滤波器,用来处理各种复杂图像。

在图像过滤中最重要的就是在保留重要的特征的同时降低噪声,这是图像滤波的目的。在很多领域上都
非常热门的神经网络训练在图像滤波上也有发展。PatchShuffle Regularization\cite{kangPatchShuffleRegularization2017}
是一种新的有关卷积神经网络(CNN)的正则化训练,它可以在每个小批量中,随机选择图像或特征图进行转换,
以便对每个局部补丁中的像素进行打乱,可以对噪声的变化更加稳健的过滤,他们在CIFAR-10上实现了5.66\%的
错误率,而不使用PatchShuffle Regularization的错误率为6.33\%。

但是在去除噪声过程不可避免会出现损坏的像素点,所以Satpathy\cite{satpathyAdaptiveNonlinearFiltering2022}
提出一种一种基于决策的非线性算法,它可以同时进行两个操作,即检测损坏的像素和评估新的像素以替换损坏的像素。在不破坏边缘和细节的情况下,实
现了对这些伪像的去除,并在噪声过大时切换到均值滤波,实现了一个算法取代多个算法,提高了效率。

图像过滤中还需要注意不能将对比度损坏,因此Balakrishnan\cite{natarajanContrastEnhancementBased2022}提出退化阈值图像检测(DTID)框架,
其使用一个快速双边过滤过程来过滤对比度图像的边缘,以改善边缘过滤图像的对比度。此方法与GUMA、HMRF、
SWT和EHS相比,DTID框架在对比度图像上花费的过滤时间减少了54\%,平均对比度增强质量提高了27\%。与最
先进的方法相比,它提高了28\%平均对比度增强质量、检测准确率26\%并且减少30\%对比度图像的过滤时间。

而滤波器中通常为单向滤波器,但是在图像处理中,图像的特征是双向的,所以在降噪过程中就会出现相位失真
而损伤图像特征。因此在图像处理中,双向滤波器是非常重要的,双向滤波器的应用非常广泛,例如图像去噪、
边缘检测和图像分割等。所以Tu\cite{tuTwoWayRecursiveFiltering2021}提出了一种双向递归滤波,此方法为了计算一个像素的过滤值,会从相邻的垂直和
水平的像素获得反馈,这样就可以保证图像的特征不会丢失,并减少过滤图像所需迭代次数。其也可以作为在深度神经网络
的一层。在实际实验中,该方法在图像去噪和边缘检测上都取得了很好的效果。

有关贝叶斯方法是一种统计框架其核心是通过新的内容更新先前的内容,其具有连贯性和灵活性,并且在图像处理上也有很好的应用。
所以有很多研究者都在研究如何将其更好的应用在图像去噪上。例如Pablo从贝叶斯推理的角度提出了一种新型的非参数降噪技术(FBADA)
\cite{sanchez-alarconFullyAdaptiveBayesian2022},它会自动改善
处理数据的信噪比,并反复评估其平滑版本,通过将信号的期望值计算为整个平滑模型集的加权平均值,就可以做到在无需进行任何参数调整就可以与标准图像
处理算法相媲美,而后者的参数已根据要恢复的真实信号进行了优化,这在实际应用中是不可能的。其对极度嘈杂(高于 20-40\%的相对误差)的信号重建得到
的残差的标准偏差可能会比原始测量的标准偏差低一个数量级以上。

在图像滤波的领域上存在复合型滤波器,他通常会结合多个滤波器的特点,例如在图像去噪中,可以结合高斯滤波器和中值滤波器,
这样就可以保证图像的平滑性和边缘的保留。所以复合型滤波器也是图像滤波的研究热点之一。而其中增强型滤波器就结和了维纳滤波器和
频谱减法技术以提高降噪和信号增强性能。首先,应用维纳滤波器估计输入信号的功率谱密度和噪声。然后,利用频谱减法技术估计噪声的功率谱密度,
从输入信号的估计功率谱密度中减去该功率谱密度。最后,再次将维纳滤波器应用于修改后的功率谱密度以产生增强信号。其高降噪性能和计算复杂度低等优点
使其应用非常广泛。所以Pranay在2019年提出了一种新的增强型滤波器\cite{kumarImageDeNoisingSalt2019}。其相对于其他滤波器对低、中密度的固定值
脉冲噪声有更好输出,该方法的主要使用修剪后的平均值取代噪声像素,以达到提高峰值信噪比(PSNR)和减少图像模糊的目的。其在实际实验中,能让视觉感更好。

自适应滤波在图像处理上是一个非常重要的研究方向,它可以根据图像的特征自动调整滤波器的参数,从而达到更好的滤波效果。
用来处理各种复杂图片,在实际应用中也是非常有用的。所以各国研究员一直在对如何改进自适应滤波器进行研究。例如Soheila提出将非参数知识结合到最小
均方自适应滤波器中(NPLMS)\cite{ashkezari-toussiIncorporatingNonparametricKnowledge2019},此方法在最大后验估计的框架下用于估计来自噪声
数据的未知参数向量,并且使用核密度估计来估计先验分布,但为了对高斯和
非高斯噪声具有鲁棒性,一些中间估计被缓冲,然后用于估计先验分布,并且不需要估计输入噪声方差,以此得出了NPLMS。此外,还提供了 NPLMS 的可变步长
版本以减少稳态误差。



% (含文献综述)
% 该部分主要需要写:
% \begin{enumerate}[label=\arabic*)]
%     \item 要实现研究目标,要使用哪些方法、系统、工具和技术?国内外研究现状、发展动态如何?
%     \item 是否还有其它方法、系统、工具和技术?分析、比较它们各自有哪些优缺点?
% \end{enumerate}
% 这就需要大家多阅读文献。关于文献还需要注意:
% \begin{enumerate}[label=\arabic*)]
%     \item 是否多为近3-5年的参考文献,且来自本领域的主流期刊?
%     \item 参考文献的引用格式是否规范?(\sout{都使用这套模板了,那肯定没问题的啦})
% \end{enumerate}
