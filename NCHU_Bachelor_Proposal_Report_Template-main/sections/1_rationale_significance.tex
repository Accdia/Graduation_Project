%%
% NCHU Bachelor Proposal Report Template
%
% 南昌航空大学毕业设计开题报告(选题的依据与意义)—— 使用 XeLaTeX 编译
%
% Copyright 2023 Arnold Chow
%
% The Current Maintainer of this work is Arnold Chow.
%
% Compile with: xelatex -> biber -> xelatex -> xelatex

\section{选题的依据及意义}

近年来,图像处理领域取得了重大进步,在医学、安全和娱乐等领域得到了广泛的应用。虽然现有的图像技术已经较为成熟,
但现有的过滤和处理图像的方法仍然存在局限性。而此时基带信号传输滤波理论为分析原始形式的信号提供了一个数学框架,
而无需进行频率调制。这种方法在信号处理中取得了成功,所以人们认为其可能可以通过将基带信号传输滤波
理论扩展到图像处理,开发出新的图像滤波方法,与现有方法相比,这些方法更精确,噪声量更小。此外,基带信号
传输滤波理论可用于分析不同图像特征之间的关系,例如纹理、颜色和形状,因此可用于开发新的图像处理算法,以更准确地
检测和保留这些特征。所以将基带信号传输滤波理论扩展到图像处理有望为图像处理和分析提供新的和改进的方法,
通过在这一领域进行研究,将有可能进一步了解如何有效和高效地处理图像,从而在使图像处理的各个领域取得进一步进展。

将基带信号传输滤波理论扩展到图像处理的研究目标是开发新的图像处理和分析方法,而为了实现此目标,则需要了解基带
信号传输滤波理论以及如何将其应用于图像处理,并清楚地了解过滤算法旨在解决的问题。这涉及确定现有方法的具体局限性
和新算法的所需特征。然后基于对问题的理解,应选择合适的理论框架。下一步是根据所选的理论框架设计算法。这涉及定义
构成算法基础的数学模型和方程。然后,设计的算法应该用合适的编程语言实现。应针对性能和效率优化实现。并通过模拟和
实验来评估新过滤算法的性能。

此研究预期成果是探索出新的可以用于缺陷检测的图像处理方法,为图像处理方法研究上提供新的研究思路。
% \textbf{前言: \LaTeX 的简单使用:}

% 在这里插入一个参考文献,仅作参考\cite{yuFeiJiZongTiDuoXueKeSheJiYouHuaDeXianZhuangYuFaZhanFangXiang2008}。
% 可以通过空一行(两次回车)实现段落换行,仅仅是回车并不会产生新的段落。

% 正文……\cite{simonyanVeryDeepConvolutional2015}

% 常使用的其他字体格式:

% {\songti \bfseries 宋体加粗} {\textbf{English}}

% {\songti \itshape 宋体斜体} {\textit{English}}

% {\songti \bfseries \itshape 宋体粗斜体} {\textbf{\textit{English}}}

% \textbf{注意!}

% 本模板必须使用 XeLaTeX + BibTeX 编译,否则会直接报错。
% 本模板支持多个平台,结合 Sublime Text/VSCode/Overleaf 都可以使用。
% 本模板支持中英文混排,但是中文和英文之间必须有空格,否则会报错。

% \vspace{10mm}

% 该部分主要需要写:
% \begin{enumerate}[label=\arabic*)]
%     \item 研究课题的名称是什么?为什么要进行该课题的研究?
%     \item 研究的目标是什么?为实现总目标,需要实现哪些子目标?
%     \item 研究的预期成果是什么?该成果具有什么科学价值?
% \end{enumerate}
