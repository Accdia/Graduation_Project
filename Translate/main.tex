\documentclass{article} %article 文档
\usepackage{anyfontsize} %设置字体大小
\usepackage{ctex}  %使用宏包(为了能够显示汉字)
\usepackage {amsmath} %数学公式
\setlength {\parindent} {0in} %段落缩进

\title{通过多模光纤同时传输射频和基带信号}  %文章标题
\author{Nishant Bindal·Rajanbir Singh Ghumaan·Prateek Jeet Singh Sohi·
 \\Nikhil Sharma·Hemdutt Joshi·Bharat Garg}%作者的名称
\date{Received:2 January 2021/Revised:23 February 2022/Accepted:9 March 2022
\\Published online:9 April 2022}       %日期

% 设置页面的环境,a4纸张大小,左右上下边距信息
\usepackage[a4paper,left=10mm,right=10mm,top=15mm,bottom=15mm]{geometry}  
\begin{document}

\maketitle          %添加这一句才能够显示标题等信息

\textbf{摘要}

数字图像处理是数字信号处理的一个子类别,重点是研究用于增强或恢复的处理技术。被各种类型的噪声破坏的图像
的去噪就属于这个类别。去噪主要是为了提高受影响图像的可理解性。用故障设备拍摄的图像或经过长途传输的图像
极易受到脉冲噪声的破坏,因此,提出了各种技术来去除图像中的这种噪声。每种技术都有自己的优点和缺点。本文
对这些技术进行了全面的比较分析,在广泛的噪声范围内进行了分析。所有的过滤技术都在MATLAB中实现,并用标准
的基准图像数据进行模拟,并对定性指标即峰值信噪比(PSNR)和结构相似度指数(SSIM)进行评估和比较。因此,
本文对各种最先进的噪声去除技术进行了全面的比较分析。
\bigskip

\textbf{关键词}
\quad 中位数滤波器,均值滤波器,盐和胡椒噪声,脉冲噪声,脉冲噪声,图像修复和去噪

\section{简介}
噪声对图像的破坏可能会导致有用的特征的损失。图像中的噪声可能是由于图像采集设备的缺陷、错误的
处理技术、远距离传输时的干扰等原因而引入的。为了从视觉上区分图像中是否存在脉冲噪声,观看者应寻找异常的亮
或暗像素。这些亮和暗的像素对应于一个像素可以获得的最大(255)和最小(0)值(在8位图像的情况下,0和255是
可能的最小和最大值)。在各种图像处理技术的帮助下,去除这些噪音的过程通常被称为去噪。\par

\hspace{2em}随着时间的推移,人们提出了各种用于恢复被盐和胡椒噪声破坏的图像的最先进的去噪滤波器。这篇
评论文章的主要目的是让读者了解各种不同的过滤器,以及这些过滤器可以被归类的各种类别的知识。这些论文的主要
贡献如下。

\begin{itemize}
\item 我们总结了大多数最先进的过滤器的工作技术。
\item 为了让读者更好地理解,还给出了使用各种技术过滤的选定基准图像(这些图像被密度为95 \% 的盐和胡椒噪声所破坏)的视觉表现。
\item 总而言之,本文将成为一个指南,使研究人员了解从1995年到2021年这四分之一世纪里盐和胡椒噪声去除领域的主要发展。
\end{itemize}

\hspace{2em}本文的其余部分组织如下。第3节解释了盐和胡椒噪声、其影响和不同的过滤技术。第4节向读者介绍了仿真
环境和对选定的过滤算法进行的比较分析。本节使用不同的表格和图表向读者介绍了这些过滤器在广泛的噪声密度范围内的
比较分析,即低度(不超过30\% )、中度(在30\%和90\%之间)和高度(超过90\%)。最后,在第5节中向读者提供了从
第4节得出的结论。

\section{评估参数}
对于各种过滤器的质量分析,使用了不同的质量评估指标。大体上,这些质量评估可以是主观的或定量的。主观的质量分析
是基于人眼从给定图像中提取结构信息的能力。然而,定量质量指标利用数学表达式来计算一个给定图像的质量。数学类很
容易实现,并提供统一的预测质量度量。在这种方法中,数学公式被应用于计算噪声/去噪声图像在原始图像上的噪声量。
在本文中,以下指标被用于质量分析。

\subsection{平均平方误差(MSE)}
平均平方误差或平均平方偏差(MSD)是处理后的像素值与准确的像素值之间的平均平方差。MSE的数学表达式由(1)给出。
\begin{equation}
    MSE= \frac{1}{M+N}\sum_{i=1}^M\sum_{i\j=1}^N(x_{i,j}-y_{i,j})^2
\end{equation}

\subsection{均方根误差(RMSE)}
RMSE可以计算为实际像素值和恢复后的像素值之间的平均平方差的根。它也可以通过取MSE的平方根来计算。
\begin{equation}
    RMSE=\sqrt{\frac{1}{M+N}\sum_{i=1}^M\sum_{i\j=1}^N(x_{i,j}-y_{i,j})^2}
\end{equation}

\subsection{峰值信噪比(PSNR)}
PSNR只是最大可能的信号功率与改变图像定量表示的噪声信号功率之间的比率。PSNR经常被用来衡量一项技术的修复精度。
PSNR可以被认为是人类对重建质量的一种近似感知。PSNR的数学表达式为分贝(dB),由(3)给出。
\begin{equation}
    PSNR=10\log_{10}({MAX^2}/{MSE})
\end{equation}
其中,MSE和MAX分别为信号的均方误差和最大值。

\subsection{结构相似性指数(SSIM)}
顾名思义,SSIM是用来测量修复/处理后的图像和实际图像之间的结构相似性。SSIM是以原始/未压缩的图像作为参考,对图
像质量进行的测量。
\begin{equation}
    SSIM=\frac{(2\mu_x\mu_y+C_1)(2\sigma_{xy}+C_2)}{(\mu_x^2+\mu_y^2+C_1)(\sigma_x^2+\sigma_y^2+C_2)}
\end{equation}
(3)至(5)中使用的术语。M, N -图像的维度

\subsection{峰值信噪比和结构相似性指数的比较}
\end{document}
