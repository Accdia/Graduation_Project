%%
% NCHU Bachelor Proposal Report Template
%
% 南昌航空大学毕业设计开题报告(国内外研究概况和发展趋势)—— 使用 XeLaTeX 编译
%
% Copyright 2023 Arnold Chow
%
% The Current Maintainer of this work is Arnold Chow.
%
% Compile with: xelatex -> biber -> xelatex -> xelatex

\section{国内外研究概况及发展趋势}
国内外关于图像滤波的研究一直是一个发展非常活跃的领域,图像滤波是图像处理中的一项关键操作,
通常在对图像进行处理时都会先对图像进行滤波,已达到消除噪声和其他伪影来提高图像质量的目的。
它具有广泛的应用,例如图像去噪、边缘检测和图像分割。因此图像滤波的方式非常多,根据不同的
处理要求有相对应的过滤技术。例如常用的高斯滤波器和中值滤波器等,这些滤波器在图像滤波上十
分有效,但因对图像的要求不断提高,这些滤波器已经达不到人们对图像的处理需求,因此各国研究员
根据不同的图像处理需求,开发出了相应的高效滤波器,用来处理各种复杂图像。

在图像过滤中最重要的就是在保留重要的特征的同时降低噪声,这是图像滤波的目的。在很多领域上都
非常热门的神经网络训练在图像滤波上也有发展。PatchShuffle Regularization\cite{kangPatchShuffleRegularization2017}
是一种新的有关卷积神经网络(CNN)的正则化训练它可以在每个小批量中,随机选择图像或特征图进行转换,
以便对每个局部补丁中的像素进行打乱,可以对噪声的变化更加稳健的过滤,他们在CIFAR-10上实现了5.66\%的
错误率,而不使用PatchShuffle Regularization的错误率为6.33\%。

但是在去除噪声过程不可避免会出现损坏的像素点,所以Satpathy\cite{satpathyAdaptiveNonlinearFiltering2022}提出一种一种基于决策的非线性算法,它可以同时
进行两个操作,即检测损坏的像素和评估新的像素以替换损坏的像素。在不破坏边缘和细节的情况下,实
现了对这些伪像的去除,并在噪声过大时切换到均值滤波,实现了一个算法取代多个算法,提高了效率。

图像过滤中还需要注意不能将对比度损坏,因此Balakrishnan\cite{natarajanContrastEnhancementBased2022}提出退化阈值图像检测(DTID)框架,
其使用一个快速双边过滤过程来过滤对比度图像的边缘,以改善边缘过滤图像的对比度。此方法与GUMA、HMRF、
SWT和EHS相比,DTID框架在对比度图像上花费的过滤时间减少了54\%,平均对比度增强质量提高了27\%。与最
先进的方法相比,它提高了28\%平均对比度增强质量、检测准确率26\%并且减少30\%对比度图像的过滤时间。

% 将基带信号传输滤波器理论扩展到图像处理的国际研究是一个活跃且不断发展的领域。在图像处理中使用滤波技术
% 作为提高图像质量和降低噪声的全新方法已经得到了广发的认可。世界各地的研究人员一直在探索在图像处理中使用
% 基带信号传输滤波理论更多的好处,目标是开发更准确、更高效和有效的方法。

% 国际研究人员开发了基于基带信号传输滤波器理论的新算法,包括降噪、边缘检测和图像恢复等方法。与现有方法相比,
% 这些方法已被证明具有更高的准确性和效率。此外,研究人员正在研究不同图像特征之间的关系以及不同类型的过滤对
% 这些特征的影响,目标是开发新的算法,在减少噪声的同时更好地保留图像特征。深度学习的兴起也影响了这一领域的
% 国际研究,研究人员探索将深度学习技术与基带信号传输滤波器理论相结合,以开发更强大的图像处理算法。综上所述,
% 国际研究界正在积极寻求开发基于基带信号传输滤波器理论的新的和改进的图像处理方法。趋势是开发比现有方法更准确、
% 更高效、更有效的算法,重点是保留重要的图像特征,同时降低噪声。

% 我国研究人员为基于基带信号传输滤波器理论的新算法的开发做出了重大贡献,包括降噪、边缘检测和图像恢复方法。
% 与现有方法相比,这些方法已被证明具有更高的准确性和效率。此外,中国研究人员正在研究不同图像特征之间的关系
% 以及不同类型的过滤对这些特征的影响,目的是开发新的算法,在减少噪声的同时更好地保留图像特征。深度学习技术
% 与基带信号传输滤波器理论的融合也是我国一个不断发展的研究领域。研究人员正在探索使用深度学习算法来开发更强大
% 的图像处理方法。综上所述,我国研究界正在积极寻求开发基于基带信号传输滤波器理论的新型和改进的图像处理方法。
% 中国研究人员正在这一领域做出重大贡献,并专注于开发比现有方法更准确、更高效、更有效的算法,特别强调在降低噪声
% 的同时保留重要的图像特征。

% (含文献综述)
% 该部分主要需要写:
% \begin{enumerate}[label=\arabic*)]
%     \item 要实现研究目标,要使用哪些方法、系统、工具和技术?国内外研究现状、发展动态如何?
%     \item 是否还有其它方法、系统、工具和技术?分析、比较它们各自有哪些优缺点?
% \end{enumerate}
% 这就需要大家多阅读文献。关于文献还需要注意:
% \begin{enumerate}[label=\arabic*)]
%     \item 是否多为近3-5年的参考文献,且来自本领域的主流期刊?
%     \item 参考文献的引用格式是否规范?(\sout{都使用这套模板了,那肯定没问题的啦})
% \end{enumerate}
